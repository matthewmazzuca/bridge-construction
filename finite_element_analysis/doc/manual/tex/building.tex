\feelchapter{Building Feel++}
            {Building Feel++}
            {Christophe Prud'homme, Baptiste Morin}
            {tutorial-building}

\section{Getting the source via an archive}
\label{sec:getting-source-via-1}

\feel is distributed as a tarball once in a while. The tarballs are available
at
\begin{center}
  \href{http://code.google.com/p/feelpp/}{http://code.google.com/p/feelpp/}
\end{center}
Download the latest tarball.

\begin{unixcom}
  tar -xzf feelpp-0.92.0.tar.gz
  cd feel-0.92.0
\end{unixcom}


\section{Getting the source via Git}
\label{sec:getting-source-via}
In order to download the sources of \feel, you can download it directly from the source depository
thanks to Git. To make it possible, you can download them anonymously or with an account in Github that you have created. As an open-source project, we strongly suggest you to create an account and take part of the project with sharing your ideas, developments or suggests. If you're interested to participate and become a \feel developer, please don't hesitate to see how it works in the appendix ~\ref{feeldevel}. For now, if you want to get the sources without an account, open a command-line and type
\begin{unixcom}
    git clone https://github.com/feelpp/feelpp.git
\end{unixcom}
then you can go to the \feel top directory with
\begin{unixcom}
		cd feel
\end{unixcom}
You should obtain furthers directories such as :
\begin{lstlisting}[language=sh]
applications/   # functional applications
benchmarks/  # applications under test
cmake/   # do not touch, used for compilation
contrib/
doc/   # tutorial and examples
feel/   # Feel++ library
ports/   # used for Mac OS X installation
research/   # research projects using Feel++
testsuite/ # Feel++ unit tests testsuite
CMakeListe.txt   # the file for cmake to build, do not modify
...
\end{lstlisting}

\section{Unix : dependencies}
\label{sec:about-dependencies}

In order to install \feel on Unix systems (other than Mac OS X, in you have a Macintosh, please go to ~\ref{macosx}), you have to install many dependencies
before. Those libraries and programs are necessary for the
compilation and installation of the \feel librairies.
This is the list of all the librairies you must have installed on your
computer, and the \lstinline|*-dev| packages for some of them. \\
\underline{Required packages}:
\begin{itemize}
\item g++ ($4.5$, $4.6$ and $4.7$) or clang++ ($\geqslant 3.1$)
\item MPI : openmpi (preferred) or mpich
\item Boost ($\geq$1.39)
\item Petsc ($\geq$2.3.3)
\item Cmake ($\geq$2.6)
\item Gmsh\footnote{Gmsh is a pre/post processing software for scientific
computing available at \url{http://www.geuz.org/gmsh}}
\item Libxml2
\end{itemize}
\underline{Optional packages}:
\begin{itemize}
\item Superlu
\item Suitesparse(umfpack)
\item Metis: scoth with the metis interface (preferred), metis (non-free)
\item Trilinos ($\geq$8.0.8)
\item Google perftools
\item Paraview\footnote{Paraview is a few parallel scientific data
    visualisation plateform, \url{http://www.paraview.org}}, this is
  not stricly required to run \feel programs but it is somehow
  necessary for visualisation
\item Python ($\geq$ 2.5) for the validation tools
\end{itemize}
Note that all these packages are available under Debian/GNU/Linux and
Ubuntu. They should be available. Once you have installed those dependencies, you can jump to ~\ref{compilingfeel}.

\section{\feel on Debian and Ubuntu}
\label{sec:feel-debian-ubuntu}

\subsection{Debian}

Debian is the platform of choice for \feel, it was developed mainly on it. The commands to install \feel on Debian are
\begin{unixcom}
  sudo apt-get update
  sudo apt-get install feel++-apps libfeel-dev feel++-doc
\end{unixcom}


The interested user is encourage to follow the \feel PTS page
\begin{itemize}
\item \feel \href{http://packages.qa.debian.org/f/feel%2B%2B.html}{Debian Packages Tracking System}
\end{itemize}

At the moment \feel compiles and is available on the following Debian
plateforms:
\begin{itemize}
\item \feel \href{https://buildd.debian.org/status/package.php?p=feel%2b%2b}{Buildd results}
\end{itemize}

\subsection{Ubuntu}

\feel was uploaded in the distribution Ubuntu-Natty (11.04) for the first
time. The commands to install \feel on Ubuntu are
\begin{unixcom}
  sudo apt-get update
  sudo apt-get install feel++-apps libfeel-dev feel++-doc
\end{unixcom}
The interested user might want to follow the Ubuntu Launchpad \feel page in
order to know what is going on with \feel on Ubuntu
\begin{itemize}
\item \feel \href{https://launchpad.net/ubuntu/+source/feel++}{Ubuntu Source
  Page for all Ubuntu versions}
\end{itemize}


\section{\Feel on Mac OS X}
\label{macosx}
\feel is also working on Mac operating systems. The way to make it work is quite different.

\subsection{Compilers}

In order to \feel and \cmake work properly, you have to install differents compilers :
\begin{itemize}
\item Gcc \\
  The first step is to install the latest version of Xcode. If your computer is
  recent, you can install it with your DVD that came with your machine (not the
  OS DVD, but the applications one). You don't have to install the complete
  Xcode (you can uncheck iOS SDK for example, it's not necessary here and
  requiers a lot of memory). Xcode will provide your computer all basic tools to
  compile such as gcc 4.2. It's the first step, you'll see later how to easily
  install gcc 4.5 or later using MacPorts.
\item Fortran \\
  To build the Makefiles, \cmake will need a Fortran compiler. To make it works,
  please go to \href{http://hpc.sourceforge.net/}{SourceForge.net} and download
  \lstinline|gfortran-snwleo-intel-bin.tar.gz| which is the fortran compiler
  only (from now, don't download the complete install with gcc 4.6 because Feel
  needs gcc 4.5 or later). To install it, go to the directory where you have
  downloaded the file and type in a command-line
\begin{unixcom}
		sudo tar -xvf gfortran-snwleo-intel-bin.tar -C /
\end{unixcom}

\end{itemize}

\subsection{MacPorts}

\paragraph{Introduction}
MacPorts is an open-source community projet which aims to design an easy-to-use
system for compiling, installing and upgrading open-source softwares on Mac OS X
operating system. It is distributed under
\href{http://opensource.org/licenses/bsd-license.php}{BSD License} and
facilitate the access to thousands of ports (softwares) without installing or
compiling open-source softwares.  MacPorts provides a single software tree which
includes the latest stable releases of approximately 8050 ports targeting the
current Mac OS X release (10.6 or 10.5). If you want more information, please
visite their \href{http://www.macports.org/}{website}.

\paragraph{Installation}
To install the latest version of MacPorts, please go to
\href{http://www.macports.org/install.php}{Installing MacPorts} page and follow
the instructions. The simplest way is to download the $dmg$ disk image
corresponding to your version of Mac OS X. It is recommended that you install
X11 (X Window System) which is normally used to display X11
applications.%, and also the Xcode Tools (only the developer tools, iOS SDK is not required).

If you have installed with the package installer
(\lstinline|MacPorts-1.x.x.dmg|) that means MacPorts will be installed in
\lstinline|/opt/local|. From now on we will suppose that macports has been
installed in \lstinline|/opt/local| which is the default MacPorts location. Note
that from now on, all tools installed by MacPorts will be installed in
\lstinline!/opt/local/bin! or \lstinline!/opt/local/sbin! for example (that's
here you'll find gcc4.5 or later e.g \lstinline!/opt/local/bin/g++-mp-4.5! once being
installed).%At the end of the installation, you can check if your PATH has been upgraded by the command \lstinline|echo $PATH| which should return a line containing \lstinline|/opt/local/bin:/opt/local/sbin|.

\paragraph{Key commands}
In your command-line, the software MacPorts is called by the command \lstinline|port|.
Here is a list of key commands for using MacPorts, if you want more informations please go to \href{http://guide.macports.org/#using.port}{MacPorts Commands}.
\begin{itemize}
\item \lstinline|sudo port -v selfupdate|
	This action should be used regularly to update the local tree with the global MacPorts ports. The option \lstinline|-v| enables verbose which generates verbose messages.
\item \lstinline|port info flowd|
	This action is used to get information about a port (description, license, maintainer, etc.)
\item \lstinline|sudo port install mypackage|
	This action install the port mypackage
\item \lstinline|sudo port uninstall mypackage|
	This action uninstall the port mypackage
\item \lstinline|port installed|
	This action displays all ports installed and their versions, variants and activation status. You can also use the \lstinline|-v| option to also display the platform and CPU architecture(s) for which the ports were built, and any variants which were explicitly negated.
\item \lstinline|sudo port upgrade mypackage|
	This action updgrades installed ports and their dependencies when a \lstinline|Portfile| in the repository has been updated. To avoid the upgrade of a port's dependencies, use the option \lstinline|-n|.
\end{itemize}

\paragraph{Portfile}
A Portfile is a TCL script which usually contains simple keyword values and TCL
expressions. Each package/port has a corresponding Portfile but it's only a part
of a port description.  \feel provides some mandatory Portfiles for its
compilation which are either not available in MacPorts or are buggy but \feel
also provides some Portfiles which are already available in MacPorts such as
gmsh or petsc. They usually provide either some fixes to ensure \feel works
properly or new version not yet available in MacPorts.  These Portfiles are
installed in \lstinline|ports/macosx/macports|.


\subsection{MacPorts and \Feel}


To be able to install \feel, add the following line in \lstinline|/opt/local/etc/macports/source.conf|
at the top of the file before any other sources :
\begin{lstlisting}[language=sh]
file:///<path to feel top directory>/ports/macosx/macports
\end{lstlisting}

Once it's done, type in a command-line :
\begin{unixcom}
		cd <your path to feel top directory>/ports/macosx/macports
		portindex -f
\end{unixcom}

You should have an output like this :
\begin{flushleft}
\fbox{
   \begin{minipage}{0.81\textwidth}
      	Reading port index in $<$your path to feel top directory$>$/ports/macosx/macports\\
	Adding port science/feel++ \\
	Adding port science/gmsh \\
	Adding port science/petsc \\ \\
	Total number of ports parsed:   3\\
	Ports successfully parsed:      3\\
	Ports failed:                   0\\
	Up-to-date ports skipped:       0
   \end{minipage}
}
\end{flushleft}
Your are now able to type
\begin{unixcom}
		sudo port install feel++
\end{unixcom}
It might take some time (possibly an entire day) to compile all the requirements for \feel
to compile properly. If you have several cores on your MacBook Pro, iMac or MacBook
we suggest that you configure macports to use all or some of them.
To do that uncomment the following line in the file  \lstinline|/opt/local/etc/macports/macports.conf|
\begin{flushleft}
\begin{lstlisting}[language=sh]
buildmakejobs	0 $\#$ all the cores
\end{lstlisting}
\end{flushleft}
At the end of the \lstinline|sudo port install feel++|, you have all
dependencies installed. To build all the Makefile, \cmake is automatically
launched but can have some libraries may not be found but they are not mandatory
for build Feel++, only the features related to the missing libraries will be
missing.

\subsection{PETSc and SLEPc on Snow Leopard and Lion}
We have heard about issues with petsc and slepc with some new MacBook Pro with
Snow Leopard while they are being installed with the command
\begin{unixcom}
  sudo port install feel++
\end{unixcom}
If it's the case, that probably means there is an issue with
atlas. If atlas is already installed, you have to unsinstall it (be careful
with dependencies, they also have to be uninstalled). Once it's done, you should
do
\begin{unixcom}
		cd <path to feel top directory>/ports/macosx/macports
		portindex -f
\end{unixcom}

then type in the exact same order :
\begin{unixcom}
		sudo port uninstall slepc
		sudo port uninstall petsc
		sudo port install -d petsc
		sudo port install slepc
\end{unixcom}
Then add to you shell script environment (e.g. for Bash shells \lstinline|.bashrc| or
\lstinline|.profile| or for CSh shells \lstinline|.tcshrc|)
\begin{unixcom}
  # Sh based shell
  export PETSC_DIR=/opt/local/lib/petsc
  export SLEPC_DIR=/opt/local/lib/petsc

  # CSh based shell
  setenv PETSC_DIR /opt/local/lib/petsc
  setenv SLEPC_DIR /opt/local/lib/petsc
\end{unixcom}
and type once again
\begin{unixcom}
		sudo port install feel++
\end{unixcom}

\noindent In that order, slepc and petsc will be installed before atlas, and feel will be properly installed.

\subsection{Missing ports}
\cmake can build Makefiles even if some packages are missing (latex2html, VTK
...). It's not necessary to install them but you can complete the installation
with MacPorts, \cmake will find them by itself once they have been installed.

\section{Compiling Feel++}
\label{compilingfeel}
Feel build system uses \cmake\index{cmake}\footnote{\url{http://www.cmake.org}}
as its build system. Check that \cmake is using gcc4.5 (or a higher version) 
or clang++ as C++ compiler
(you can use the option \lstinline|CMAKE_CXX_COMPILER=<path>/g++-4.5| where the
\lstinline|path| depends on your OS, it's probably \lstinline|/usr/bin| or
\lstinline|/opt/local/bin| but you can also change it with the command \lstinline|ccmake|
and press \lstinline|t| for advanced options).  
\feel, using \cmake, can be built either in source and out of source and different
build type:
\begin{itemize}
\item minsizerel : minimal size release
\item release release
\item debug : debug
\item none(default)
\end{itemize}

\paragraph{CMake Out Source Build (preferred)}
The best way is to have a directory (\lstinline|FEEL| for example) in which you have : \\
\begin{lstlisting}
	feel/
\end{lstlisting}
where \lstinline|feel| is the top directory where the source have been downloaded. Placed in \lstinline|FEEL|, you can create the build directory (\lstinline|feel.opt| for example) and lauch cmake with :
\begin{unixcom}
  mkdir feel.opt
  cd feel.opt
  cmake <directory where the feel source are>
  # e.g cmake ../feel if feel.opt is at the same
  # directory level as feel
\end{unixcom}
you can customize the build type:
\begin{unixcom}
  # Choose g++ release
  cmake -D CMAKE_CXX_COMPILER=/usr/bin/g++-4.5
  # Debug build type (-g...)
  cmake -D CMAKE_BUILD_TYPE=Debug
  # Release build type (-O3...)
  cmake -D CMAKE_BUILD_TYPE=Release
  ...
\end{unixcom}
Once Cmake has made its work, you are now able to compile the library with
\begin{unixcom}
		make
\end{unixcom}
\textbf{Important :} from now, all commands should be type in \lstinline|feel.opt| or its subdirectories.


% \subsection{Compiling Feel with the AutoTools}

% Go in the same folder in wich you have done the checkout and type the following
% commands :

% \subsubsection{From tarball}
% \label{sec:from-tarball}

% The steps are as follows to configure the \feel Development Plateform

% \begin{unixcom}
%   cd feel-x.y.z
%   mkdir opt
%   cd opt

%   ../configure --enable-opt2
% \end{unixcom}

% Then type
% \begin{unixcom}
%   make
% \end{unixcom}

% to compile the library and the tutorial. And finally type
% \begin{unixcom}
%   make check
% \end{unixcom}
% In order to compile the testsuite, the examples and the benchmarks and
% execute some of them to verity that the \feel library is functional.


% \subsubsection{From subversion}
% \label{sec:from-subversion}

% The steps are as follows to configure the \feel Development Plateform

% \begin{unixcom}
%   cd feel
%   make -f Makefile.dist
%   mkdir opt
%   cd opt

%   ../configure --enable-opt2
% \end{unixcom}

% Then, to build  the \feel library, type
% \begin{unixcom}
%   make
% \end{unixcom}

% And finally to check the library, type
% \begin{unixcom}
%   make check
% \end{unixcom}


% \begin{note}
%   The script \lstinline!configure! supports many command line
%   options. In particular if you are interested in writing some code or
%   examples inside the \feel environment you have to enable the so
%   called \emph{maintainer mode} to ensure that the makefiles are
%   properly regenerated when you modify a \lstinline!Makefile.am! or if
%   you modify \lstinline!configure.ac!, to achieve this type
%   \begin{unixcom}
%     configure --enable-maintainer-mode
%   \end{unixcom}
%   To list all configure options, type
%   \begin{unixcom}
%     configure --help
%   \end{unixcom}
% \end{note}


% \subsubsection{Compiling an extra module}
% \label{sec:compile-an-extra}

% If you work with an extra module, \emph{e.g.} \lstinline!validation!, the steps are as follows
% \begin{unixcom}
% cd feel
% make -f Makefile.dist
% cd benchmarks/validation
% make -f Makefile.dist
% cd ../../..
% mkdir opt
% cd opt
% ../feel/configure --enable-opt2 --enable-maintainer-mode
% make
% make check
% \end{unixcom}


\subsection{Compiling the Feel++ manual}
\label{sec:comp-feel-tutor}
The manual (which includes the tutorial) is edited with \LaTeX  so you need to have installed the \LaTeX  distribution on your computer. \LaTeX  is a high-quality typesetting system, it includes features designed for the production of technical and scientific documentation. There are several ways to make it work, for example you can go on \href{http://www.tug.org/mactex/}{MacTeX website} and follow the instructions to install the distribution. If the command \lstinline|make check| in \lstinline|feel.opt/| has been run before, the tutorial
should be already compiled and ready. The steps are as follows to build the Feel tutorial
\begin{unixcom}
  cd feel.opt/doc/manual
  make pdf
\end{unixcom}
%Here is what the directory should look like
%\begin{unixcom}
%  cd opt/doc/tutorial
%  ls
%
%  laplacian     Makefile      myintegrals   mymesh       pngs/
%  tutorial.blg  tutorial.out  tutorial.toc  laplacian.o  myapp
%  myintegrals.o mymesh.o      stokes.assert tutorial.aux pdfs/ styles/
%  stokes        stokes.o      tutorial.bbl  tutorial.log tutorial.pdf
%\end{unixcom}
The directory \lstinline|doc/manual| contains all examples used in the tutorial. You will see how it works in the following parts.

%%% Local Variables:
%%% coding: utf-8
%%% mode: latex
%%% TeX-PDF-mode: t
%%% TeX-parse-self: t
%%% x-symbol-8bits: nil
%%% TeX-auto-regexp-list: TeX-auto-full-regexp-list
%%% TeX-master: "../feel-manual"
%%% ispell-local-dictionary: "american"
%%% End:

